%%%%%%%% DATA LITERACY 2025 LATEX PROJECT TEMPLATE FILE %%%%%%%%%%%%%%%%%
%%% Based on the 2025 ICML template, available at https://icml.cc/Conferences/2025/AuthorInstructions %%%

\documentclass{article}

% Recommended, but optional, packages for figures and better typesetting:
\usepackage{microtype}
\usepackage{graphicx}
\usepackage{subfigure}
\usepackage{booktabs} % for professional tables

\usepackage{tikz}
% Corporate Design of the University of Tübingen
% Primary Colors
\definecolor{TUred}{RGB}{165,30,55}
\definecolor{TUgold}{RGB}{180,160,105}
\definecolor{TUdark}{RGB}{50,65,75}
\definecolor{TUgray}{RGB}{175,179,183}

% Secondary Colors
\definecolor{TUdarkblue}{RGB}{65,90,140}
\definecolor{TUblue}{RGB}{0,105,170}
\definecolor{TUlightblue}{RGB}{80,170,200}
\definecolor{TUlightgreen}{RGB}{130,185,160}
\definecolor{TUgreen}{RGB}{125,165,75}
\definecolor{TUdarkgreen}{RGB}{50,110,30}
\definecolor{TUocre}{RGB}{200,80,60}
\definecolor{TUviolet}{RGB}{175,110,150}
\definecolor{TUmauve}{RGB}{180,160,150}
\definecolor{TUbeige}{RGB}{215,180,105}
\definecolor{TUorange}{RGB}{210,150,0}
\definecolor{TUbrown}{RGB}{145,105,70}

% hyperref makes hyperlinks in the resulting PDF.
% If your build breaks (sometimes temporarily if a hyperlink spans a page)
% please comment out the following usepackage line and replace
% \usepackage{icml2023} with \usepackage[nohyperref]{icml2023} above.
\usepackage{hyperref}


% Attempt to make hyperref and algorithmic work together better:
\newcommand{\theHalgorithm}{\arabic{algorithm}}

\usepackage[accepted]{icml2025}

% For theorems and such
\usepackage{amsmath}
\usepackage{amssymb}
\usepackage{mathtools}
\usepackage{amsthm}
\usetikzlibrary{arrows.meta, positioning, calc}

% if you use cleveref..
\usepackage[capitalize,noabbrev]{cleveref}

% Todonotes is useful during development; simply uncomment the next line
%    and comment out the line below the next line to turn off comments
%\usepackage[disable,textsize=tiny]{todonotes}
\usepackage[textsize=tiny]{todonotes}


% The \icmltitle you define below is probably too long as a header.
% Therefore, a short form for the running title is supplied here:
\icmltitlerunning{Exposure-Normalized Bicycle Crash Risk Along Berlin Routes} % I think our title is not too long, hence we can use the same as for \icmltitle

\begin{document}

\twocolumn[
\icmltitle{Exposure-Normalized Bicycle Crash Risk Along Berlin Routes}

% It is OKAY to include author information, even for blind
% submissions: the style file will automatically remove it for you
% unless you've provided the [accepted] option to the icml2023
% package.

% List of affiliations: The first argument should be a (short)
% identifier you will use later to specify author affiliations
% Academic affiliations should list Department, University, City, Region, Country
% Industry affiliations should list Company, City, Region, Country

% You can specify symbols, otherwise they are numbered in order.
% Ideally, you should not use this facility. Affiliations will be numbered
% in order of appearance and this is the preferred way.
\icmlsetsymbol{equal}{*}

\begin{icmlauthorlist}
\icmlauthor{Eric Berger}{equal}
\icmlauthor{Edward Eichhorn}{equal}
\icmlauthor{Liaisan Faidrakhmanova}{equal}
\icmlauthor{Luise Grasl}{equal}
\icmlauthor{Tobias Schnarr}{equal}
\end{icmlauthorlist}

% fill in your matrikelnummer, email address, degree, for each group member
% \icmlaffiliation{first}{Matrikelnummer 7064584, MSc Machine Learning}
% \icmlaffiliation{second}{Matrikelnummer 12345678, MSc Quantitative Data Science}
% \icmlaffiliation{third}{Matrikelnummer 7320172, MSc Quantitative Data Science}
% \icmlaffiliation{fourth}{Matrikelnummer 7329274, MSc Quantitative Data Science}
% \icmlaffiliation{fith}{Matrikelnummer 7304640, MSc Quantitative Data Science}

% put your email addresses here. You can use initials to save space, 
% e.g. if you are called Max Mustermann, you can use \icmlcorrespondingauthor{MM}{max.mustermann@uni-tuebingen.de}
% DO USE YOUR UNIVERSITY EMAIL ADDRESS!
% \icmlcorrespondingauthor{EB}{eric.berger@student.uni-tuebingen.de} 
% \icmlcorrespondingauthor{EE}{first2.last2@uni-tuebingen.de}
% \icmlcorrespondingauthor{LF}{liaisan.faidrakhmanova@student.uni-tuebingen.de}
% \icmlcorrespondingauthor{LG}{luise.grasl@student.uni-tuebingen.de}
\icmlcorrespondingauthor{TS}{tobias-marco.schnarr@student.uni-tuebingen.de}

% You may provide any keywords that you
% find helpful for describing your paper; these are used to populate
% the "keywords" metadata in the PDF but will not be shown in the document
\icmlkeywords{Machine Learning, ICML}

\vskip 0.3in
]

% this must go after the closing bracket ] following \twocolumn[ ...

% This command actually creates the footnote in the first column
% listing the affiliations and the copyright notice.
% The command takes one argument, which is text to display at the start of the footnote.
% The \icmlEqualContribution command is standard text for equal contribution.
% Remove it (just {}) if you do not need this facility.

%\printAffiliationsAndNotice{}  % leave blank if no need to mention equal contribution
\printAffiliationsAndNotice{\icmlEqualContribution} % otherwise use the standard text.

\begin{abstract}
We investigate bicycle crash risk on Berlin’s urban street network, addressing a key limitation of many safety analyses: raw crash counts conflate danger with demand and fail to distinguish 
intrinsically risky locations from high-use roads. We combine police-reported crashes with a city-wide dataset of measured bicycle volumes to compute exposure-adjusted risk at the street-segment 
and junction levels. Risk estimates, available at monthly resolution, can be conditioned on contextual factors (e.g., time of day, weather) to capture temporal variation. Aggregating risk to 
arbitrary routes enables comparisons that trade off safety against convenience. The result is a reproducible framework for context-aware, exposure-controlled bicycle safety analysis and routing.\todo{rewrite/edit, when we have final results}
\end{abstract}

\section{Introduction}\label{sec:intro}
\begin{figure*}[t]
  \centering
  \includegraphics{figs/map_3_panels.pdf}
  \caption{\textbf{Safety-aware routing pipeline for the Berlin cycling network.}
  Panels (a–c) show an example month (June 2021) and are zoomed in for readability; see \cref{sec:methods} for formal definitions and notation.
  \textbf{(a):} police-recorded bicycle crashes ({\color{TUorange}points}) and street segments with measured cyclist exposure ({\color{TUgray}lines}, used as the base network in all panels).
  \textbf{(b):} monthly segment risk (accidents per 10{,}000 trips); high-risk segments in {\color{TUred}red} are the top 90th percentile of \(r_{s,t}\) in the displayed month. {\color{TUdark}Circles} mark junctions (degree $\ge 3$), for which we also estimate junction risk from exposure aggregated over incident segments.
  \textbf{(c):} shortest path ({\color{TUblue}blue}) versus a safer alternative ({\color{TUgreen}green}) chosen to reduce route risk under a distance-detour constraint. {\color{TUocre}Filled circle} and {\color{TUocre}cross} denote origin and destination, respectively; {\color{TUdark}circles} show junctions for reference.
  }
  \label{fig:visual_app}
\end{figure*}

Cycling safety analyses often rely on raw crash counts, which conflate danger with demand: streets that attract many cyclists tend to accumulate more incidents even when per-rider risk
is low~\citep{luecken2018}. This obscures intrinsically risky locations and limits both targeted interventions and everyday route choice, especially in dense urban networks such as Berlin~\citep{Uijtdewilligen01092024}. 
We address this by estimating exposure-adjusted crash risk on the street network via a relative-risk formulation that separates cyclist demand from intrinsic danger and remains stable under 
sparse or unevenly distributed observations. The resulting estimates are transformed into expected crash costs and propagated to route-level scores suitable for navigation. Our study combines 
police-recorded crashes from the German Unfallatlas (Berlin subset)~\citep{Unfallatlas2025} with a city-wide dataset of measured bicycle volumes at the street-segment level~\citep{kaiser_2025_15332147}.
We compute exposure-adjusted relative risk for individual street segments and, motivated by the concentration of crashes at intersections, derive junction-level risk by aggregating volumes 
from adjoining segments. Risk is estimated at monthly resolution and can be conditioned on contextual factors such as time of day and weather to capture temporal variability. We then aggregate 
segment- and junction-level expected crash costs to evaluate arbitrary routes and compare alternatives that reduce estimated risk while maintaining comparable convenience in terms of distance 
or travel time, see \cref{fig:visual_app}. We address the question: how can exposure-adjusted relative crash risk be estimated from measured bicycle volumes and integrated into context-aware
routing? Our work makes the following contributions: (i) a reproducible pipeline for estimating exposure-adjusted relative crash risk at street and junction levels from measured cyclist volumes, 
with support for context-conditional analysis; and (ii) a route-scoring method that integrates network-level risk into safety-aware routing under convenience constraints.

\paragraph{Related work.}
To avoid conflating danger with demand, prior work normalizes bicycle crashes by cyclist exposure~\citep{luecken2018}. City-scale studies show that exposure-normalized risk yields more 
informative spatial patterns than raw counts and that finer temporal resolution improves inference, while noting persistent under-reporting in police records~\citep{Uijtdewilligen01092024}. 
A central challenge is obtaining reliable exposure: some approaches extrapolate city-wide volumes from sparse counters using learning-based models and multi-source features, with short 
measurement campaigns improving predictions at new locations~\citep{Kaiser_Klein_Kaack_2025}, whereas more recent efforts provide street-segment datasets of measured bicycle volumes, enabling 
downstream safety analyses without modeling exposure~\citep{kaiser2025spatiotemporalgraphneuralnetwork}. At the network level, studies define risk as crashes per unit exposure on links and 
address practical issues such as spatial snapping of crashes, allocating events near intersections, and integrating safety metrics into routing under convenience constraints~\citep{isprs-archives-XLIII-B4-2022-427-2022}. 
Intersection safety is critical: case analyses document strong crash concentrations at junctions and stress controlling for exposure when comparing infrastructure types or locations~\citep{futuretransp1030037}. 
We estimate exposure-adjusted crash risk for Berlin at street-segment and junction levels, enable context-conditional risk analysis, and integrate network-level risk into routing decisions.

The paper is organized as follows: we present our data in \cref{sec:data}, describe our methods in \cref{sec:methods}, report the results in \cref{sec:results}, and conclude in \cref{sec:conclusion}.

\section{Data}\label{sec:data}
% In this section, describe \emph{what you did}. Roughly speaking, explain what data you worked with, how or from where it was collected, it's structure and size. Explain your analysis, and any specific choices you made in it. Depending on the nature of your project, you may focus more or less on certain aspects. If you collected data yourself, explain the collection process in detail. If you downloaded data from the net, show an exploratory analysis that builds intuition for the data, and shows that you know the data well. If you are doing a custom analysis, explain how it works and why it is the right choice. If you are using a standard tool, it may still help to briefly outline it. Cite relevant works. You can use the \verb|\citep| and \verb|\citet| commands for this purpose \citep{mackay2003information}.
We combine police-recorded bicycle crashes with measured cyclist exposure for the city of Berlin. Crash data are drawn from the Berlin subset of the German \emph{Unfallatlas}~\citep{Unfallatlas2025} 
and filtered for bicycle-related incidents. Exposure comes from a city-wide dataset of measured bicycle volumes aggregated at the street-segment level~\citep{kaiser2025spatiotemporalgraphneuralnetwork}. 
The street network is represented as polyline segments with associated monthly cyclist counts. The resulting dataset spans 2019--2023 and covers 4{,}958 street segments and 2{,}924 junctions, 
with 33{,}181 recorded bicycle crashes. At monthly resolution the data are sparse: in a typical month fewer than 5\% of segments and about 3\% of junctions record at least one crash, and 
some periods include segments with zero measured exposure.

To enable network-scale analysis, all layers are harmonized to a common street-network topology and a projected coordinate reference system. Crash locations are matched to segments using 
nearest-segment assignment. To capture the concentration of crashes at intersections, we identify junctions as nodes where at least three segments meet; crashes within a fixed radius are 
assigned to the nearest junction, and junction exposure is computed from the exposures of incident segments. Matched crashes and exposures are aggregated to monthly resolution, and segments 
and months with zero exposure are dropped. To obtain stable risk estimates under sparse observations, monthly aggregates are pooled to yearly totals and an Empirical Bayes 
approach is applied as described in \cref{sec:methods}. The resulting yearly segment- and junction-level risk estimates serve as inputs to all routing analyses.

\section{Methods}\label{sec:methods}
\paragraph{Empirical Bayes relative risk.}
For each street segment $s$ and month $t$, let $A_{s,t}$ denote the number of police-recorded bicycle crashes and $E_{s,t}$ the measured cyclist exposure. We assume that, within a given month, crash incidence is proportional to exposure under a ``no special risk'' baseline, yielding the expected number of crashes
\[
\widehat{A}_{s,t}
= A_{t}\,\frac{E_{s,t}}{E_{t}},
\qquad
A_{t}=\sum_{s} A_{s,t},\ \ E_{t}=\sum_{s} E_{s,t}.
\]

Although crashes and exposure are aggregated at monthly resolution in preprocessing, sparse observations motivate estimating relative risk at yearly resolution. For each year $y$ comprising months $\mathcal{T}(y)$, we pool monthly aggregates to yearly totals
\[
A_{s,y}=\sum_{t\in\mathcal{T}(y)} A_{s,t},
\qquad
E_{s,y}=\sum_{t\in\mathcal{T}(y)} E_{s,t},
\]
and define $A_y=\sum_s A_{s,y}$ and $E_y=\sum_s E_{s,y}$. The corresponding baseline expectation is
\[
\widehat{A}_{s,y}
= A_y\,\frac{E_{s,y}}{E_y}.
\]

To obtain stable estimates under sparse observations, we introduce a latent relative-risk multiplier $\theta_{s,y}$ and model
\[
\begin{aligned}
A_{s,y} \mid \theta_{s,y} &\sim \text{Poisson}(\widehat{A}_{s,y}\,\theta_{s,y}), \\
\theta_{s,y} &\sim \text{Gamma}(\alpha,\alpha),
\end{aligned}
\]
where the Gamma distribution is parameterized in shape--rate form, enforcing a unit prior mean
$\mathbb{E}[\theta_{s,y}]=1$. The hyperparameter $\alpha$ is estimated via empirical Bayes by maximizing the marginal likelihood pooled across all segments and years. By conjugacy, the posterior mean
\[
r_{s,y}
= \mathbb{E}[\theta_{s,y}\mid A_{s,y},\widehat{A}_{s,y}]
= \frac{A_{s,y}+\alpha}{\widehat{A}_{s,y}+\alpha}
\]
serves as our smoothed segment-level relative risk. This estimator shrinks extreme values toward $1$, with stronger shrinkage for segments with low expected crash counts.

To capture the concentration of crashes near intersections, we also estimate junction-level relative risk. A junction is defined as any network node with degree $\ge 3$; crashes within a fixed radius of its centroid are assigned to that junction. Because a traversal typically contributes exposure to two incident segments (entering and exiting), we approximate junction exposure by a half-sum of incident segment exposures,
\begin{equation}
\label{eq:half-sum}
E_{v,y}=\tfrac{1}{2}\sum_{s\in\mathcal{I}(v)}E_{s,y},
\end{equation}
which avoids double-counting. Aggregating exposure over incident links is common when turning movements are unavailable~\citep{hakkert2002uses,WANG2020105838}. The expected number of junction crashes is defined as $\widehat{A}_{v,y}=A_y E_{v,y}/E_y$, and junction relative risk $r_{v,y}$ is estimated using the same Empirical Bayes formulation and hyperparameters as for segments. All risks are estimated independently by year.

\paragraph{Routing graph.}
For each year $y$, we build an undirected graph $G_y=(V_y,E_y)$ from the street network: nodes represent segment endpoints and edges represent street segments with length $\ell_e$. Each graph edge $e$ corresponds to a street segment $s$ and inherits its estimated yearly risk, denoted $r_{e,y}=r_{s,y}$. Junction identifiers and risks are mapped to nodes via spatial snapping in a projected coordinate system, yielding a year-specific, risk-annotated network.

\paragraph{Safety-aware routing.}
We compare shortest-distance routes with alternatives that reduce estimated crash risk under a bounded detour. For year $y$, the length of a route $P$ is
\[
L(P)=\sum_{e\in P}\ell_e.
\]
To account for both segment- and junction-level exposure, the risk contribution of edge $e=(u,v)$ is defined as
\[
\rho_{e,y}=r_{e,y}+\eta\,\frac{r_{u,y}+r_{v,y}}{2},
\]
where $r_{u,y}$ and $r_{v,y}$ are junction risks (set to zero for non-junction nodes) and $\eta\ge 0$ weights the contribution of junction risk. We interpret these relative-risk weights as an \emph{additive surrogate} for cumulative route risk, not as probabilities.

Given an origin--destination pair, the baseline route $P_{\text{dist}}$ minimizes $L(P)$. The safety-aware route solves
\begin{equation}
\label{eq:safe-routing}
\begin{aligned}
P_{\text{safe}}=\arg\min_{P}\ & R(P)=\sum_{e\in P}\rho_{e,y} \\
\text{s.t.}\ & L(P)\le (1+\varepsilon)\,L(P_{\text{dist}}),
\end{aligned}
\end{equation}
where $\varepsilon$ is the allowable relative detour. We approximate this constrained problem using a weighted-sum sweep: for $\lambda\in\Lambda$,
\[
P(\lambda)=\arg\min_{P}\sum_{e\in P}\bigl(\rho_{e,y}+\lambda\,\ell_e\bigr),
\]
then select among feasible candidates the route with minimal $R(P)$, breaking ties by shorter $L(P)$. Shortest paths are computed using Dijkstra’s algorithm. Routing is performed independently by year.

\paragraph{Evaluation metrics.}
For each origin--destination pair, we report the relative length increase
\[
\Delta_L=\frac{L(P_{\text{safe}})-L(P_{\text{dist}})}{L(P_{\text{dist}})}
\]
and the relative risk reduction
\[
\Delta_R=\frac{R(P_{\text{dist}})-R(P_{\text{safe}})}{R(P_{\text{dist}})}.
\]
Pairs with $R(P_{\text{dist}})=0$ are excluded from $\Delta_R$ due to an undefined denominator. These metrics quantify how bounded detours trade distance for reductions in relative crash risk.

% How to make a figure
% \begin{figure}[ht]
% \vskip 0.2in
% \begin{center}
% \centerline{\includegraphics[width=\columnwidth]{figures/teacher-student.pdf}}
% \caption{Historical locations and number of accepted papers for International
% Machine Learning Conferences (ICML 1993 -- ICML 2008) and International
% Workshops on Machine Learning (ML 1988 -- ML 1992). At the time this figure was
% produced, the number of accepted papers for ICML 2008 was unknown and instead
% estimated.}
% \label{icml-historical}
% \end{center}
% \vskip -0.2in
% \end{figure}

\section{Results}\label{sec:results}
% In this section outline your results. At this point, you are just stating the outcome of your analysis. You can highlight important aspects (``we observe a significantly higher value of $x$ over $y$''), but leave interpretation and opinion to the next section. This section absoultely \emph{must} include at least two figures.

\section{Discussion and Conclusion}\label{sec:conclusion}
% Use this section to briefly summarize the entire text. Highlight limitations and problems, but also make clear statements where they are possible and supported by the analysis. 
We provide implementation details, hyperparameters, and supplementary material, available at \url{https://github.com/ytobiaz/data_literacy}.

\newpage

% \section*{Contribution Statement}
% Explain here, in one sentence per person, what each group member contributed. For example, you could write: Max Mustermann collected and prepared data. Gabi Musterfrau and John Doe performed the data analysis. Jane Doe produced visualizations. All authors will jointly wrote the text of the report. Note that you, as a group, a collectively responsible for the report. Your contributions should be roughly equal in amount and difficulty.

% \section*{Notes} 
% Your entire report has a \textbf{hard page limit of 4 pages} excluding references and the contribution statement. (I.e. any pages beyond page 4 must only contain the contribution statement and references). Appendices are \emph{not} possible. But you can put additional material, like interactive visualizations or videos, on a githunb repo (use \href{https://github.com/pnkraemer/tueplots}{links} in your pdf to refer to them). Each report has to contain \textbf{at least three plots or visualizations}, and \textbf{cite at least two references}. More details about how to prepare the report, inclucing how to produce plots, cite correctly, and how to ideally structure your github repo, will be discussed in the lecture, where a rubric for the evaluation will also be provided.


\bibliography{bibliography}
\bibliographystyle{icml2025}

\end{document}

% This document was modified from the files available at https://icml.cc/Conferences/2025/AuthorInstructions
% the full copyright notice is available within the file icml2025.sty