%%%%%%%% DATA LITERACY 2025 LATEX PROJECT TEMPLATE FILE %%%%%%%%%%%%%%%%%
%%% Based on the 2025 ICML template, available at https://icml.cc/Conferences/2025/AuthorInstructions %%%

\documentclass{article}

% Recommended, but optional, packages for figures and better typesetting:
\usepackage{microtype}
\usepackage{graphicx}
\usepackage{subfigure}
\usepackage{booktabs} % for professional tables

\usepackage{tikz}
% Corporate Design of the University of Tübingen
% Primary Colors
\definecolor{TUred}{RGB}{165,30,55}
\definecolor{TUgold}{RGB}{180,160,105}
\definecolor{TUdark}{RGB}{50,65,75}
\definecolor{TUgray}{RGB}{175,179,183}

% Secondary Colors
\definecolor{TUdarkblue}{RGB}{65,90,140}
\definecolor{TUblue}{RGB}{0,105,170}
\definecolor{TUlightblue}{RGB}{80,170,200}
\definecolor{TUlightgreen}{RGB}{130,185,160}
\definecolor{TUgreen}{RGB}{125,165,75}
\definecolor{TUdarkgreen}{RGB}{50,110,30}
\definecolor{TUocre}{RGB}{200,80,60}
\definecolor{TUviolet}{RGB}{175,110,150}
\definecolor{TUmauve}{RGB}{180,160,150}
\definecolor{TUbeige}{RGB}{215,180,105}
\definecolor{TUorange}{RGB}{210,150,0}
\definecolor{TUbrown}{RGB}{145,105,70}
\definecolor{PNorange}{RGB}{255,153,51}

% hyperref makes hyperlinks in the resulting PDF.
% If your build breaks (sometimes temporarily if a hyperlink spans a page)
% please comment out the following usepackage line and replace
% \usepackage{icml2023} with \usepackage[nohyperref]{icml2023} above.
\usepackage{hyperref}


% Attempt to make hyperref and algorithmic work together better:
\newcommand{\theHalgorithm}{\arabic{algorithm}}

% command for plotting legend colorbar of risk map figure
\newcommand{\coolwarmbox}{%
  \tikz[baseline]{
    \begin{scope}[yshift=0.1ex]
      \draw[black!70]
        (0,0) rectangle (4em,1ex);
      \shade[
        shading=axis,
        shading angle=0,
        left color=TUblue,
        middle color= TUlightblue,
        right color=TUlightblue
      ]
      (0,0) rectangle (1em,1ex);
      \shade[
        shading=axis,
        shading angle=0,
        left color=TUlightblue,
        middle color=TUgray,
        right color=TUgray
      ]
      (0.9em,0) rectangle (2em,1ex);
      \shade[
        shading=axis,
        shading angle=0,
        left color=TUgray,
        middle color=PNorange,
        right color=PNorange
      ]
      (1.9em,0) rectangle (3em,1ex);
      \shade[
        shading=axis,
        shading angle=0,
        left color=PNorange,
        middle color=TUred,
        right color=TUred
      ]
      (2.9em,0) rectangle (4em,1ex);
    \end{scope}
  }%
}



\usepackage[accepted]{icml2025}

% For theorems and such
\usepackage{amsmath}
\usepackage{amssymb}
\usepackage{mathtools}
\usepackage{amsthm}
\usetikzlibrary{arrows.meta, positioning, calc}

% if you use cleveref..
\usepackage[capitalize,noabbrev]{cleveref}

% Todonotes is useful during development; simply uncomment the next line
%    and comment out the line below the next line to turn off comments
%\usepackage[disable,textsize=tiny]{todonotes}
\usepackage[textsize=tiny]{todonotes}


% The \icmltitle you define below is probably too long as a header.
% Therefore, a short form for the running title is supplied here:
\icmltitlerunning{Exposure-Adjusted Bicycle Crash Risk Estimation and Safer Routing in Berlin} % I think our title is not too long, hence we can use the same as for \icmltitle

\begin{document}

\twocolumn[
\icmltitle{Exposure-Adjusted Bicycle Crash Risk Estimation and Safer Routing in Berlin}

% It is OKAY to include author information, even for blind
% submissions: the style file will automatically remove it for you
% unless you've provided the [accepted] option to the icml2023
% package.

% List of affiliations: The first argument should be a (short)
% identifier you will use later to specify author affiliations
% Academic affiliations should list Department, University, City, Region, Country
% Industry affiliations should list Company, City, Region, Country

% You can specify symbols, otherwise they are numbered in order.
% Ideally, you should not use this facility. Affiliations will be numbered
% in order of appearance and this is the preferred way.
\icmlsetsymbol{equal}{*}

\begin{icmlauthorlist}
\icmlauthor{Eric Berger}{equal}
\icmlauthor{Edward Eichhorn}{equal}
\icmlauthor{Liaisan Faidrakhmanova}{equal}
\icmlauthor{Luise Grasl}{equal}
\icmlauthor{Tobias Schnarr}{equal}
\end{icmlauthorlist}

% fill in your matrikelnummer, email address, degree, for each group member
% \icmlaffiliation{first}{Matrikelnummer 7064584, MSc Machine Learning}
% \icmlaffiliation{second}{Matrikelnummer 12345678, MSc Quantitative Data Science}
% \icmlaffiliation{third}{Matrikelnummer 7320172, MSc Quantitative Data Science}
% \icmlaffiliation{fourth}{Matrikelnummer 7329274, MSc Quantitative Data Science}
% \icmlaffiliation{fith}{Matrikelnummer 7304640, MSc Quantitative Data Science}

% put your email addresses here. You can use initials to save space, 
% e.g. if you are called Max Mustermann, you can use \icmlcorrespondingauthor{MM}{max.mustermann@uni-tuebingen.de}
% DO USE YOUR UNIVERSITY EMAIL ADDRESS!
% \icmlcorrespondingauthor{EB}{eric.berger@student.uni-tuebingen.de} 
% \icmlcorrespondingauthor{EE}{first2.last2@uni-tuebingen.de}
% \icmlcorrespondingauthor{LF}{liaisan.faidrakhmanova@student.uni-tuebingen.de}
% \icmlcorrespondingauthor{LG}{luise.grasl@student.uni-tuebingen.de}
\icmlcorrespondingauthor{Tobias Schnarr}{tobias-marco.schnarr@student.uni-tuebingen.de}

% You may provide any keywords that you
% find helpful for describing your paper; these are used to populate
% the "keywords" metadata in the PDF but will not be shown in the document
\icmlkeywords{Machine Learning, ICML}

\vskip 0.3in
]

% this must go after the closing bracket ] following \twocolumn[ ...

% This command actually creates the footnote in the first column
% listing the affiliations and the copyright notice.
% The command takes one argument, which is text to display at the start of the footnote.
% The \icmlEqualContribution command is standard text for equal contribution.
% Remove it (just {}) if you do not need this facility.

%\printAffiliationsAndNotice{}  % leave blank if no need to mention equal contribution
\printAffiliationsAndNotice{\icmlEqualContribution} % otherwise use the standard text.

\begin{abstract}
Accurately estimating the risk of bicycle crashes at street level requires consideration of both crash counts and cyclist exposure. However, exposure data from official counting stations is unavailable for most streets. This makes it difficult to identify streets that are dangerous. We therefore use Strava's bike 
trip data to estimate the relative crash risk across street segments and junctions in Berlin. We identify those with a higher or lower than expected occurrence of crashes, and enable a routing algorithm to suggest lower-risk routes.

\end{abstract}

\section{Introduction}\label{sec:intro}
\begin{figure*}[t]
  \centering
  \includegraphics{figs/map_3_panels.pdf}
  \caption{\textbf{Safety-aware routing pipeline for the Berlin network.}
  Panels (a–c) are zoomed in for readability; see \cref{sec:methods} for definitions and notation.
  (a) Police-recorded bicycle crashes in June 2021 ({\color{TUorange}points}) and street segments with measured cyclist exposure ({\color{TUgray}lines}).
  (b) Pooled segment-level relative crash risk estimated from all available data; high-risk segments in {\color{TUred}red} correspond to values above the 90th percentile of relative risk; {\color{TUdark}circles} mark junctions (degree $\ge 3$).
  (c) Shortest path ({\color{TUblue}blue}) versus a safer alternative ({\color{TUgreen}green}) selected to reduce cumulative relative route risk under a route-length constraint. {\color{TUocre}Filled circle} and {\color{TUocre}cross} denote origin and destination, respectively.
  }
  \label{fig:visual_app}
\end{figure*}

Cycling is far from a safe endeavour. 92,882 bicycle crashes were recorded in 2024, including 441 fatalities - 16\% of all traffic deaths that year ~\citep{Unfallatlas2025}. 
Yet it is rarely clear which streets are most dangerous and thus should be avoided by cyclists or made safer. 
Quantifying street-level danger is non-trivial because simple crash counts confound risk with exposure. 
Streets with high exposure, i.e. high numbers of cyclists, tend to accumulate more crashes even when per-cyclist risk is low~\citep{luecken2018}. Crashes must be normalised by cyclist counts; otherwise, dangerous streets can remain hidden in dense urban networks~\citep{Uijtdewilligen01092024}. 
Unfortunately, street-level cyclist counts are rare. Berlin, for example, provides hourly counts via official counting stations, but their limited coverage (20 stations for thousands of streets) makes them impractical for city-wide risk estimation~\citep{BerlinZaehlen_Fahrradbarometer}.
We address this problem by using bike trip counts from the fitness-tracking app Strava. These have been used to predict official bike counts~\citep{dadashova2020estimation}. We show that they can serve as a proxy for cyclist exposure and estimate for all segments and junctions in Berlin’s official cycling network relative risks (the ratio of observed to expected crashes).
Because Strava coverage can be sparse, we use empirical Bayes smoothing for estimation~\citep{clayton1987empirical}. This stabilises estimates for low-exposure segments and junctions, and quantifies uncertainty. We also introduce a routing algorithm that finds substantially lower-risk routes under a route-length constraint.


\section{Data}\label{sec:data}
Multiple datasets were used for risk estimation. Crash counts were taken from the \emph{German Accident Atlas}~\citep{Unfallatlas2025}, which provides geodata of police-reported crashes where people were injured. We filtered the data to bicycle-related crashes within the city limits of Berlin. Cyclist exposure was approximated using the dataset by \citet{kaiser2025spatiotemporalgraphneuralnetwork}, 
which reports daily street-segment-level counts of bicycle trips recorded via the Strava app in Berlin from 2019 to 2023. Strava users are not representative of the general cycling population (they skew younger, male, and sport-oriented; \citealp{kaiser2025spatiotemporalgraphneuralnetwork}). Therefore, we assess potential bias by comparing segment-level count shares in 2023 with official bicycle 
counter data from the city of Berlin ~\citep{BerlinRadverkehrzaehlstellen2023} for the subset of segments where both Strava and official counts are available (Figure~\ref{fig:segment_share}). Count shares correlate strongly ($r = .61$) and are preserved in the Strava data. Segments on main streets where you can ride fast (e.g., Karl-Marx-Allee) are overrepresented in the Strava data, since those are more often tracked. Residential streets (e.g., Kollwitzstraße) are underrepresented, 
since slower, everyday cycling is less often tracked.
All datasets were combined into one dataframe and matched to the same street network. The network is represented as segments with associated monthly exposure counts. We map crashes to the network using nearest-segment assignment. Junctions are defined as nodes where at least three segments meet and crashes within a fixed radius are assigned to the nearest junction. Junction exposure is derived from the segment exposure (see \cref{sec:methods}). 
At monthly resolution, events are sparse: in a typical month, fewer than 5\% of segments and 3\% of junctions record at least one crash. We drop segments with zero recorded trips over at least one year and pool counts over the full period 2019--2023 for risk estimation. The dataset comprises 4{,}335 segments, 2{,}862 junctions, and 15{,}396 recorded bicycle crashes. 

\begin{figure}[ht]
  \centering
  \includegraphics{figs/segment_share.pdf}
  \caption{Consistency check between official bicycle counts and Strava bike trips at the street-segment level (2023). Points show segment-wise shares of total annual counts.}
  \label{fig:segment_share}
\end{figure}

\section{Methods}\label{sec:methods}
\paragraph{Crash, exposure, and risk measures.}
For each month $t$, let $C_{s,t}$ and $E_{s,t}$ denote the number of police-recorded bicycle crashes and measured cyclist exposure on street segment $s$. Junction crashes $C_{v,t}$ are defined as crashes within a fixed radius of junction $v$. Because a traversal typically contributes exposure to two incident segments, we approximate junction exposure by the half-sum of incident segment exposures,
\[
E_{v,t}=\tfrac{1}{2}\sum_{s\in\mathcal I(v)} E_{s,t},
\]
a common approach when turning movements are unavailable~\citep{hakkert2002uses,WANG2020105838}.
For notational convenience, both street segments and junctions are indexed by a generic entity index $i$, with $A_{i,t}$ and $E_{i,t}$ denoting the corresponding crash and exposure quantities.

Under a no-special-risk baseline, monthly crash incidence is assumed proportional to exposure, yielding the expected number of crashes
\[
\widehat{C}_{i,t}
= C_{\cdot t}\,\frac{E_{i,t}}{E_{\cdot t}},
\qquad
C_{\cdot t}=\sum_i C_{i,t},\ \ E_{\cdot t}=\sum_i E_{i,t},
\]
where sums are taken jointly over all segments and junctions, defining a shared baseline.
Because routing requires a pooled baseline risk estimate, crashes and baseline expectations are aggregated over the full period,
\[
C_i=\sum_t C_{i,t},
\qquad
\widehat{C}_i=\sum_t \widehat{C}_{i,t},
\]
and the raw relative risk $r^{\text{raw}}_i$ is $C_i/\widehat{C}_i$.

\paragraph{Empirical Bayes smoothing.}
Because many segments have low exposure and thus very small expected counts, the raw relative risk is highly variable. That is why we use Empirical Bayes smoothing to improve the risk estimates.
This methods shrinks low-exposure estimates toward a baseline, while high-exposure estimates change little.
Concretely, we assume a true relative risk  $r^{\text{true}}_i$ such that the observed count $C_i$ follows a Poisson model with 
\[
C_i \mid r^{\text{true}}_i \sim \text{Poisson}(\widehat{C}_i\,r^{\text{true}}_i).
\]
The Poisson distribution is natural for nonnegative event counts over a fixed time period under a baseline rate, and it yields $\mathbb{E}[C_i]=\widehat C_i$ when $r^{\text{true}}_i=1$. To allow heterogeneity in 
relative risk beyond this baseline, we place a Gamma prior on $r^{\text{true}}_i$ in the shape–rate parameterization,

\[
r^{\text{true}}_i \sim \text{Gamma}(\alpha,\alpha),
\]
which enforces $\mathbb{E}[r^{\text{true}}_i]=1$ and has variance $\mathrm{Var}(r^{\text{true}}_i)=1/\alpha$ controlling the amount of shrinkage. The Gamma prior is also conjugate 
to the Poisson likelihood, giving a closed-form posterior
\[
r^{\text{true}}_i \mid C_i,\widehat{C}_i \sim \text{Gamma}(C_i+\alpha,\;\widehat{C}_i+\alpha),
\]
so posterior inference is simple and numerically stable. We estimate $\alpha$ from the data using method of moments~\citep{Morris1983}, as

\[
\widehat{\alpha}
=
\frac{\sum_i \widehat{C}_i^{\,2}}
{\sum_i (C_i-\widehat{C}_i)^2 - \sum_i \widehat{C}_i}.
\]

and use the posterior mean
\[
\widehat{r}_i=\mathbb{E}[r^{\text{true}}_i\mid C_i,\widehat C_i]=\frac{C_i+\alpha}{\widehat C_i+\alpha}
\]
as the smoothed relative risk. For small $\widehat C_i$, $r_i$ is pulled toward 1, while for large $\widehat C_i$ it approaches the raw ratio $C_i/\widehat C_i$. Uncertainty is summarized by $(1-\delta=0.95)$ 
equal-tailed credible intervals from quantiles of the Gamma posterior.

\paragraph{Risk-weighted routing graph.}
Relative risk estimates are dimensionless and conditional on exposure. To obtain additive routing weights, we rescale relative risk by the pooled baseline crash rate,
\[
\bar{\lambda}=\frac{C_\cdot}{E_\cdot},
\qquad
C_\cdot=\sum_i C_i,\ \ E_\cdot=\sum_i E_i,
\]
yielding the routing weight
\[
w_i=\bar{\lambda}\,r_i.
\]

We construct an undirected graph $G=(V,E)$ from the street network, where nodes correspond to segment endpoints and edges to street segments of length $\ell_e$. Each edge $e$ corresponds to a 
segment $s$ and inherits its weight, $w_e=w_s$. Junction identifiers and weights are mapped to nodes via spatial snapping in a projected coordinate system, producing a single risk-annotated 
network.

\paragraph{Safety-aware routing.}
We compare shortest-distance routes with alternatives that reduce estimated crash risk under a bounded detour. The length of a route $P$ is
\[
L(P)=\sum_{e\in P}\ell_e.
\]
To incorporate segment- and junction-level risk, the risk contribution of edge $e=(u,v)$ is defined as
\[
\rho_e = w_e + \eta\,\frac{w_u+w_v}{2},
\]
where $w_u$ and $w_v$ denote junction routing weights (zero for non-junction nodes), yielding an additive surrogate for cumulative route risk.

For an origin--destination pair, the baseline route $P_{\text{dist}}$ minimizes $L(P)$. The safety-aware route is obtained by solving
\begin{equation}
\label{eq:safe-routing}
\begin{aligned}
P_{\text{safe}}=\arg\min_{P}\ & R(P)=\sum_{e\in P}\rho_e \\
\text{s.t.}\ & L(P)\le (1+\varepsilon)\,L(P_{\text{dist}}),
\end{aligned}
\end{equation}
where $\varepsilon$ is the allowable relative detour~\citep{ehrgott2005multicriteria}. We approximate this constraint using a weighted-sum sweep: for $\lambda\in\Lambda$,
\[
P(\lambda)
=\arg\min_{P}
\left(
\sum_{e\in P}\rho_e
+\lambda\sum_{e\in P}\ell_e
\right),
\]
and select the feasible route minimizing $R(P)$. Shortest paths are computed using Dijkstra’s algorithm~\citep{Dijkstra1959}.

\paragraph{Evaluation metrics.}
For each origin--destination pair, we report the relative length increase
\[
\Delta_L=\frac{L(P_{\text{safe}})-L(P_{\text{dist}})}{L(P_{\text{dist}})}
\]
and the relative risk reduction
\[
\Delta_R=\frac{R(P_{\text{dist}})-R(P_{\text{safe}})}{R(P_{\text{dist}})}.
\]
Pairs with $R(P_{\text{dist}})=0$ are excluded from $\Delta_R$. We additionally report the expected number of avoided crashes,
\[
\Delta_C = R(P_{\text{dist}})-R(P_{\text{safe}}).
\]

\begin{figure*}[ht]
  \centering
  \includegraphics{figs/risk_3_panel.pdf}
  \caption{\textbf{Risk heatmap and detailed inspection of junction 2482.}
  The colors \coolwarmbox\space in panels (a)--(b) indicate $\log_{10}$-scaled risk values, ranging from low risk {\color{TUblue}(-2)} to high risk {\color{TUred}(2)}.
  (a) Section of the Strava bike network in Berlin with all computed road segments and junctions displayed.
  (b) Closer view of junction 2482 and the crashes (black dots) assigned to it, with risk values shown using the same color scale.
  (c) Street-level view of junction 2482 \citep{streetview2024}, providing visual context for the observed risk.
  }
  \label{fig:junction_details}
\end{figure*}


\section{Related Work}\label{sec:relatedw}

Prior studies combined police crash records with exposure estimates, but early work extrapolated bicycle traffic from motorized transport data, poorly capturing actual cycling patterns \citep{isprs-archives-XLIII-B4-2022-427-2022}. Other work calibrated crowdsourced GPS cycling data with count stations and showed that cyclist volumes predict crash risk, though uncertainty of risk estimates on low-volume segments remained unexamined \citep{Uijtdewilligen01092024, futuretransp1030037}.

Where direct counts are unavailable, supervised learning and graph neural network approaches estimate city-wide volumes by combining sparse counters with crowdsourced data \citep{Kaiser_Klein_Kaack_2025, kaiser2025spatiotemporalgraphneuralnetwork}. While these models required manual validation counts for calibration, they demonstrated that Strava data correlates with official counting stations at segment level, supporting its use as an exposure proxy.

To account for accident sparsity, studies employed Poisson-Gamma count models \citep{luecken2018}, but these remained limited to city-level aggregation, lacking segment-level estimates needed for routing applications. 

\section{Results}\label{sec:results}
As described in \cref{sec:methods}, relative risk $r_i$ was computed for each segment and junction.  
Due to high variance in the crash data, the shrinkage parameter was small ($\widehat{\alpha} = 0.129$), resulting in limited 
regularization and a wide spread of $r_i$ estimates. These were mapped as partly shown in \cref{fig:junction_details}(a).  
Most elements exhibit low risk: 64.4\% of segments and 69.1\% of junctions lie within the confidence bounds ($r_i = 1$).  
Values of $r_i<1$ occur for 17.6\% of segments and 25.8\% of junctions, while 17.9\% of segments and 4.9\% of junctions show elevated risk ($r_i>1$).  
Risk values range from 0.03-50.79 for segments and 0.03-6.43 for junctions.  
No overlap exists between the ten segments with the highest relative risk and those with the highest number of crashes.  
To verify that the method identifies high-risk locations, one such site was examined in detail.  
At junction 2482 ($r_i = 6.43$), 22 crashes occurred despite moderate traffic.  
All involved at least one additional vehicle - mostly cars (20) - and mainly resulted from turning or crossing maneuvers.  
As shown in \cref{fig:junction_details}(c), car lanes intersect the bicycle lane at this junction.

We evaluated the algorithm using 1,000 random origin–destination pairs, comparing the shortest-path baseline against safest alternatives across a range of allowable detour and junction-risk constraints ~\citep{NateraOrozco2020}.

\begin{table}[ht]
\centering
\small
\caption{Trade-off between path length increase and safety improvement under varying detour budgets ($\varepsilon$).
Values are aggregated over all origin--destination pairs and reported as medians.
The junction penalty weight is fixed to $\eta = 1$.
$\Delta_L$ denotes the relative path length increase,
$\Delta_R$ the relative reduction in expected crashes with respect to the shortest-path baseline,
and $\Delta_C$ the absolute number of avoided expected crashes per 100{,}000 trips, reported as rounded integer counts.}
\label{tab:routing_tradeoff}
\begin{tabular}{@{}ccccc@{}}
\toprule
$\varepsilon$
& $\Delta_L\,(\mathrm{IQR})$
& $\Delta_R\,(\mathrm{IQR})$
& $\Delta_R>0$
& $\Delta_C\,(\mathrm{IQR})$ \\
\midrule
0.00 & 0.000\,(0.000) & 0.000\,(0.000) & 0.037 & 0\,(0) \\
0.05 & 0.007\,(0.022) & 0.101\,(0.210) & 0.767 & 39\,(123) \\
0.10 & 0.015\,(0.037) & 0.147\,(0.240) & 0.841 & 61\,(132) \\
0.15 & 0.024\,(0.063) & 0.169\,(0.239) & 0.873 & 71\,(141) \\
0.20 & 0.041\,(0.109) & 0.192\,(0.238) & 0.901 & 83\,(150) \\
0.30 & 0.091\,(0.156) & 0.237\,(0.228) & 0.928 & 104\,(157) \\
0.40 & 0.137\,(0.188) & 0.262\,(0.230) & 0.944 & 112\,(170) \\
0.50 & 0.165\,(0.200) & 0.281\,(0.222) & 0.948 & 121\,(178) \\
\bottomrule
\end{tabular}
\end{table}

\cref{tab:routing_tradeoff} summarizes the trade-off between route length and exposure-adjusted crash risk under bounded detours.
With a strictly bounded detour of only 5\%, the median risk reduction ranges between 18.5\% and 24.6\%, depending on the junction penalty.
Larger detours further increase these gains, reaching median risk reductions of 38\%-43\% at $\varepsilon=0.2$.
Safer route alternatives are available for the vast majority of trips (76\%–92\%) within the given detour budget.
As the budget increases to 20\%, over 90\% of routes have a feasible, lower-risk alternative.
Across all detour budgets, increasing $\eta$ is associated with lower median risk reductions.


\section{Discussion and Conclusion}\label{sec:conclusion}
% Use this section to briefly summarize the entire text. Highlight limitations and problems, but also make clear statements where they are possible and supported by the analysis. 
We developed a city-wide method for estimating bicycle crash risk using Strava exposure data. The approach enables safety-aware routing that balances crash risk reduction against route length constraints.

Our approach faces two main limitations. First, official crash data capture only personal injury incidents and suffer from under-reporting, systematically underestimating absolute risk. Second, Strava data skew toward recreational rides on main roads, underrepresenting casual trips on residential streets. Where Strava overcounts exposure, calculated risk is too low; where it undercounts, sparse data yield inflated estimates. Empirical Bayes smoothing partially mitigates this by shrinking unstable low-exposure estimates toward the citywide baseline.

Our junction exposure model assumes uniform flow and ignores turning movements. This misallocates exposure where one main road dominates side streets. Systematic biases in crowdsourced exposure data limit risk estimates' accuracy. While graph neural network methods can interpolate traffic volumes from sparse data \citep{kaiser2025spatiotemporalgraphneuralnetwork}, they require manual validation counts. Cities with sparse counting networks like Berlin could address this by expanding automated counting infrastructure.

Despite these limitations, the approach scales to any city with crash records and crowdsourced trip data, requiring only modest detours for substantially safer routes.
\clearpage

\section*{Contribution Statement}
Explain here, in one sentence per person, what each group member contributed. For example, you could write: Max Mustermann collected and prepared data. Gabi Musterfrau and John Doe performed the data analysis. Jane Doe produced visualizations. All authors will jointly wrote the text of the report. Note that you, as a group, a collectively responsible for the report. Your contributions should be roughly equal in amount and difficulty.

% \section*{Notes} 
% Your entire report has a \textbf{hard page limit of 4 pages} excluding references and the contribution statement. (I.e. any pages beyond page 4 must only contain the contribution statement and references). Appendices are \emph{not} possible. But you can put additional material, like interactive visualizations or videos, on a githunb repo (use \href{https://github.com/pnkraemer/tueplots}{links} in your pdf to refer to them). Each report has to contain \textbf{at least three plots or visualizations}, and \textbf{cite at least two references}. More details about how to prepare the report, inclucing how to produce plots, cite correctly, and how to ideally structure your github repo, will be discussed in the lecture, where a rubric for the evaluation will also be provided.


\bibliography{bibliography}
\bibliographystyle{icml2025}

\end{document}

% This document was modified from the files available at https://icml.cc/Conferences/2025/AuthorInstructions
% the full copyright notice is available within the file icml2025.sty