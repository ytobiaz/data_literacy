%%%%%%%% DATA LITERACY 2025 LATEX PROJECT TEMPLATE FILE %%%%%%%%%%%%%%%%%
%%% Based on the 2025 ICML template, available at https://icml.cc/Conferences/2025/AuthorInstructions %%%

\documentclass{article}

% Recommended, but optional, packages for figures and better typesetting:
\usepackage{microtype}
\usepackage{graphicx}
\usepackage{subfigure}
\usepackage{booktabs} % for professional tables

\usepackage{tikz}
% Corporate Design of the University of Tübingen
% Primary Colors
\definecolor{TUred}{RGB}{165,30,55}
\definecolor{TUgold}{RGB}{180,160,105}
\definecolor{TUdark}{RGB}{50,65,75}
\definecolor{TUgray}{RGB}{175,179,183}

% Secondary Colors
\definecolor{TUdarkblue}{RGB}{65,90,140}
\definecolor{TUblue}{RGB}{0,105,170}
\definecolor{TUlightblue}{RGB}{80,170,200}
\definecolor{TUlightgreen}{RGB}{130,185,160}
\definecolor{TUgreen}{RGB}{125,165,75}
\definecolor{TUdarkgreen}{RGB}{50,110,30}
\definecolor{TUocre}{RGB}{200,80,60}
\definecolor{TUviolet}{RGB}{175,110,150}
\definecolor{TUmauve}{RGB}{180,160,150}
\definecolor{TUbeige}{RGB}{215,180,105}
\definecolor{TUorange}{RGB}{210,150,0}
\definecolor{TUbrown}{RGB}{145,105,70}

% hyperref makes hyperlinks in the resulting PDF.
% If your build breaks (sometimes temporarily if a hyperlink spans a page)
% please comment out the following usepackage line and replace
% \usepackage{icml2023} with \usepackage[nohyperref]{icml2023} above.
\usepackage{hyperref}


% Attempt to make hyperref and algorithmic work together better:
\newcommand{\theHalgorithm}{\arabic{algorithm}}

% command for plotting legend colorbar of risk map figure
\newcommand{\coolwarmbox}{%
  \tikz[baseline]{
    \begin{scope}[yshift=0.1ex]
      \draw[black!70]
        (0,0) rectangle (4em,1ex);
      \shade[
        shading=axis,
        shading angle=0,
        left color=TUblue,
        middle color= TUgray,
        right color=TUgray
      ]
      (0,0) rectangle (2em,1ex);
      \shade[
        shading=axis,
        shading angle=0,
        left color=TUgray,
        middle color=TUgray,
        right color=TUred
      ]
      (1.9em,0) rectangle (4em,1ex);
    \end{scope}
  }%
}


\usepackage[accepted]{icml2025}

% For theorems and such
\usepackage{amsmath}
\usepackage{amssymb}
\usepackage{mathtools}
\usepackage{amsthm}
\usetikzlibrary{arrows.meta, positioning, calc}

% if you use cleveref..
\usepackage[capitalize,noabbrev]{cleveref}

% Todonotes is useful during development; simply uncomment the next line
%    and comment out the line below the next line to turn off comments
%\usepackage[disable,textsize=tiny]{todonotes}
\usepackage[textsize=tiny]{todonotes}


% The \icmltitle you define below is probably too long as a header.
% Therefore, a short form for the running title is supplied here:
\icmltitlerunning{Exposure-Adjusted Bicycle Crash Risk Estimation and Safer Routing in Berlin} % I think our title is not too long, hence we can use the same as for \icmltitle

\begin{document}

\twocolumn[
\icmltitle{Exposure-Adjusted Bicycle Crash Risk Estimation and Safer Routing in Berlin}

% It is OKAY to include author information, even for blind
% submissions: the style file will automatically remove it for you
% unless you've provided the [accepted] option to the icml2023
% package.

% List of affiliations: The first argument should be a (short)
% identifier you will use later to specify author affiliations
% Academic affiliations should list Department, University, City, Region, Country
% Industry affiliations should list Company, City, Region, Country

% You can specify symbols, otherwise they are numbered in order.
% Ideally, you should not use this facility. Affiliations will be numbered
% in order of appearance and this is the preferred way.
\icmlsetsymbol{equal}{*}

\begin{icmlauthorlist}
\icmlauthor{Eric Berger}{equal}
\icmlauthor{Edward Eichhorn}{equal}
\icmlauthor{Liaisan Faidrakhmanova}{equal}
\icmlauthor{Luise Grasl}{equal}
\icmlauthor{Tobias Schnarr}{equal}
\end{icmlauthorlist}

% fill in your matrikelnummer, email address, degree, for each group member
% \icmlaffiliation{first}{Matrikelnummer 7064584, MSc Machine Learning}
% \icmlaffiliation{second}{Matrikelnummer 12345678, MSc Quantitative Data Science}
% \icmlaffiliation{third}{Matrikelnummer 7320172, MSc Quantitative Data Science}
% \icmlaffiliation{fourth}{Matrikelnummer 7329274, MSc Quantitative Data Science}
% \icmlaffiliation{fith}{Matrikelnummer 7304640, MSc Quantitative Data Science}

% put your email addresses here. You can use initials to save space, 
% e.g. if you are called Max Mustermann, you can use \icmlcorrespondingauthor{MM}{max.mustermann@uni-tuebingen.de}
% DO USE YOUR UNIVERSITY EMAIL ADDRESS!
% \icmlcorrespondingauthor{EB}{eric.berger@student.uni-tuebingen.de} 
% \icmlcorrespondingauthor{EE}{first2.last2@uni-tuebingen.de}
% \icmlcorrespondingauthor{LF}{liaisan.faidrakhmanova@student.uni-tuebingen.de}
% \icmlcorrespondingauthor{LG}{luise.grasl@student.uni-tuebingen.de}
\icmlcorrespondingauthor{Tobias Schnarr}{tobias-marco.schnarr@student.uni-tuebingen.de}

% You may provide any keywords that you
% find helpful for describing your paper; these are used to populate
% the "keywords" metadata in the PDF but will not be shown in the document
\icmlkeywords{Machine Learning, ICML}

\vskip 0.3in
]

% this must go after the closing bracket ] following \twocolumn[ ...

% This command actually creates the footnote in the first column
% listing the affiliations and the copyright notice.
% The command takes one argument, which is text to display at the start of the footnote.
% The \icmlEqualContribution command is standard text for equal contribution.
% Remove it (just {}) if you do not need this facility.

%\printAffiliationsAndNotice{}  % leave blank if no need to mention equal contribution
\printAffiliationsAndNotice{\icmlEqualContribution} % otherwise use the standard text.

\begin{abstract}
Accurately estimating the risk of bicycle crashes at street level requires consideration of both crash counts and cyclist exposure. However, exposure data from official counting stations is unavailable for most sections of the street network. This makes it difficult to identify the streets that are dangerous and should be avoided and prioritised for safety improvements. We therefore use Strava's bike 
trip data to estimate the relative crash risk across street segments and junctions in Berlin. These risk estimates identify those with a higher or lower than expected occurrence of crashes, and enable a routing algorithm to suggest lower-risk routes.

\end{abstract}

\section{Introduction}\label{sec:intro}
\begin{figure*}[t]
  \centering
  \includegraphics{figs/map_3_panels.pdf}
  \caption{\textbf{Safety-aware routing pipeline for the Berlin network.}
  Panels (a–c) are zoomed in for readability; see \cref{sec:methods} for definitions and notation.
  (a) Police-recorded bicycle crashes in June 2021 ({\color{TUorange}points}) and street segments with measured cyclist exposure ({\color{TUgray}lines}).
  (b) Pooled segment-level relative crash risk estimated from all available data; high-risk segments in {\color{TUred}red} correspond to values above the 90th percentile of relative risk; {\color{TUdark}circles} mark junctions (degree $\ge 3$).
  (c) Shortest path ({\color{TUblue}blue}) versus a safer alternative ({\color{TUgreen}green}) selected to reduce cumulative relative route risk under a distance-detour constraint. {\color{TUocre}Filled circle} and {\color{TUocre}cross} denote origin and destination, respectively.
  }
  \label{fig:visual_app}
\end{figure*}

Cycling is far from a safe endeavour. 92,882 bicycle crashes were recorded in 2024, including 441 fatalities - 16\% of all traffic deaths that year ~\citep{Unfallatlas2025}. 
Yet it is rarely clear which streets are most dangerous and thus should be avoided by cyclists or prioritised for safety improvements. 

Quantifying street-level danger is non-trivial because simple crash counts confound risk with exposure. 
Streets with high exposure, i.e. high numbers of cyclists, tend to accumulate more crashes even when per-cyclist risk is low~\citep{luecken2018}. To reveal high-risk locations, crashes must be normalised by cyclist counts; otherwise, dangerous streets can remain hidden in dense urban networks~\citep{Uijtdewilligen01092024}. 
Unfortunately, comprehensive street-level cyclist counts are rarely available. Berlin, for example, provides hourly counts at selected locations via official counting stations, but their limited spatial coverage (20 stations for thousands of streets) makes them impractical for city-wide risk estimation~\citep{BerlinZaehlen_Fahrradbarometer}.

We address this problem by using bike trip counts from the fitness-tracking app Strava. These have been used to predict official counting-station data~\citep{dadashova2020estimation} and we show that they can serve as a proxy for cyclist exposure. For all segments and junctions in Berlin’s official cycling network, we estimate exposure-normalised relative risk, defined as the ratio of observed to expected crashes.
Because Strava coverage can be sparse, we use empirical Bayes smoothing for estimation~\citep{clayton1987empirical}, which stabilises estimates in low-exposure segments and junctions, and quantifies uncertainty. Additionally, 
building on these estimates, we introduce a routing algorithm that finds substantially lower-risk alternatives under a route-length constraint.


\section{Data}\label{sec:data}
Multiple datasets were used for risk estimation. Crash counts were taken from the \emph{German Accident Atlas}~\citep{Unfallatlas2025}, which provides georeferenced locations of police-reported crashes where people were injured. We filtered the data to bicycle-related crashes within the city limits of Berlin. Cyclist exposure was approximated using the dataset by \citet{kaiser2025spatiotemporalgraphneuralnetwork}, 
which reports daily street-segment-level counts of bicycle trips recorded via the Strava app in Berlin from 2019 to 2023. Strava users are not representative of the general cycling population (they skew younger, male, and sport-oriented; \citealp{kaiser2025spatiotemporalgraphneuralnetwork}). Therefore, we assess potential bias by comparing segment-level count shares in 2023 with official bicycle 
counter data from the city of Berlin ~\citep{BerlinRadverkehrzaehlstellen2023} for the subset of segments where both Strava data and official counts are available (Figure~\ref{fig:segment_share}). Count shares correlate strongly ($r = .61$) and are overall well preserved in the Strava data. Segments on wide main streets (e.g., Karl-Marx-Allee) are overrepresented in the Strava data, likely reflecting faster rides that are more often tracked, whereas residential streets (e.g., Kollwitzstraße) are underrepresented, 
consistent with slower, local cycling that is less often tracked.

\begin{figure}[ht]
  \centering
  \includegraphics{figs/segment_share.pdf}
  \caption{Consistency check between official bicycle counts and Strava-derived cyclist volumes at the street-segment level (2023). Points show segment-wise shares of total annual counts; the dashed 
  line denotes equality between the two measures.}
  \label{fig:segment_share}
\end{figure}

All datasets were combined into one dataframe and matched to the same street network and map projection. The network is represented as polyline segments with associated monthly exposure counts. We map crashes to the network using nearest-segment assignment. Junctions are defined as nodes where at least three segments meet and crashes within a fixed radius are assigned to the nearest junction. Junction exposure is derived from the segment exposure (see \cref{sec:methods} for details). 

At monthly resolution, events are sparse: in a typical month, fewer than 5\% of segments and 3\% of junctions record at least one crash. We drop segments with zero recorded trips over at least one year and pool counts over the full period 2019--2023 for risk estimation. The dataset comprises 4{,}335 segments, 2{,}862 junctions, and 15{,}396 recorded bicycle crashes. The resulting segment- and junction-level relative risk estimates are used throughout the analysis and serve as inputs to the routing algorithm.

\section{Methods}\label{sec:methods}
\paragraph{Empirical Bayes relative risk and uncertainty.}
For each month $t$, let $A_{s,t}$ and $E_{s,t}$ denote the number of police-recorded bicycle crashes and measured cyclist exposure on street segment $s$. Junction crashes $A_{v,t}$ are defined as crashes within a fixed radius of junction $v$. Because a traversal typically contributes exposure to two incident segments, we approximate junction exposure by the half-sum of incident segment exposures,
\[
E_{v,t}=\tfrac{1}{2}\sum_{s\in\mathcal I(v)} E_{s,t},
\]
a common approach when turning movements are unavailable~\citep{hakkert2002uses,WANG2020105838}.
For notational convenience, both street segments and junctions are indexed by a generic entity index $i$, with $A_{i,t}$ and $E_{i,t}$ denoting the corresponding crash and exposure quantities.

Under a no-special-risk baseline, monthly crash incidence is assumed proportional to exposure, yielding the expected number of crashes
\[
\widehat{A}_{i,t}
= A_{\cdot t}\,\frac{E_{i,t}}{E_{\cdot t}},
\qquad
A_{\cdot t}=\sum_i A_{i,t},\ \ E_{\cdot t}=\sum_i E_{i,t},
\]
where sums are taken jointly over all segments and junctions, defining a shared baseline.
Because routing requires a pooled baseline risk estimate, crashes and baseline expectations are aggregated over the full study period,
\[
A_i=\sum_t A_{i,t},
\qquad
\widehat{A}_i=\sum_t \widehat{A}_{i,t},
\]
and the raw relative risk is $A_i/\widehat{A}_i$.

To stabilize estimation under sparsity, we introduce a latent relative-risk multiplier $\theta_i$ and model
\[
A_i \mid \theta_i \sim \text{Poisson}(\widehat{A}_i\,\theta_i),
\qquad
\theta_i \sim \text{Gamma}(\alpha,\alpha),
\]
where the Gamma prior (shape--rate) enforces $\mathbb{E}[\theta_i]=1$~\citep{Lord2010}. 
The hyperparameter $\alpha$ is estimated from the data, such that inference on $\theta_i$ is performed using Empirical Bayes smoothing.

Under the Poisson--Gamma model,
\[
\mathbb{E}[A_i] = \widehat{A}_i
\qquad\text{and}\qquad
\mathrm{Var}(A_i) = \widehat{A}_i + \frac{\widehat{A}_i^{\,2}}{\alpha}.
\]
The shrinkage parameter $\alpha$ is estimated via the method of moments~\citep{Morris1983} by equating empirical and theoretical second moments across entities,
\[
\widehat{\alpha}
=
\frac{\sum_i \widehat{A}_i^{\,2}}
{\sum_i (A_i-\widehat{A}_i)^2 - \sum_i \widehat{A}_i}.
\]

Inference for entity-specific risk follows directly from Bayes' theorem,
\[
p(\theta_i \mid A_i,\widehat{A}_i)
\;\propto\;
p(A_i \mid \theta_i,\widehat{A}_i)\,p(\theta_i),
\]
which, by conjugacy of the Poisson likelihood and Gamma prior, yields the posterior distribution
\begin{equation}
\label{eq:gamma_dist}
\theta_i \mid A_i,\widehat{A}_i
\sim \text{Gamma}(A_i+\alpha,\;\widehat{A}_i+\alpha).
\end{equation}
The posterior mean
\[
r_i
= \mathbb{E}[\theta_i \mid A_i,\widehat{A}_i]
= \frac{A_i+\alpha}{\widehat{A}_i+\alpha}
\]
serves as the Empirical Bayes relative risk, with stronger shrinkage toward the baseline for entities with low expected counts~\citep{Hauer2002}. Uncertainty is quantified using $(1-\delta)$ credible intervals obtained from quantiles of the Gamma posterior.

\paragraph{Risk-weighted routing graph.}
Relative risk estimates are dimensionless and conditional on exposure. To obtain additive weights for routing, we rescale relative risk by the pooled baseline crash rate
\[
\bar{\lambda}=\frac{A_\cdot}{E_\cdot},
\qquad
A_\cdot=\sum_i A_i,\ \ E_\cdot=\sum_i E_i.
\]
The resulting routing weight is
\[
w_i=\bar{\lambda}\,r_i.
\]
We construct an undirected graph $G=(V,E)$ from the street network. Nodes correspond to segment endpoints and edges to street segments with length $\ell_e$. Each edge $e$ corresponds to a segment $s$ 
and inherits its routing weight $w_e=w_s$. Junction identifiers and weights are mapped to nodes via spatial snapping in a projected coordinate system, yielding a single risk-annotated network.

\paragraph{Safety-aware routing.}
We compare shortest-distance routes with alternatives that reduce estimated crash risk under a bounded detour. The length of a route $P$ is
\[
L(P)=\sum_{e\in P}\ell_e.
\]
To account for segment- and junction-level risk, the risk contribution of edge $e=(u,v)$ is
\[
\rho_e = w_e + \eta\,\frac{w_u+w_v}{2},
\]
where $w_u$ and $w_v$ are junction routing weights (zero for non-junction nodes) and $\eta\ge 0$ controls the contribution of junction risk. These quantities form an \emph{additive surrogate} for cumulative route risk.

For an origin--destination pair, the baseline route $P_{\text{dist}}$ minimizes $L(P)$. The safety-aware route solves
\begin{equation}
\label{eq:safe-routing}
\begin{aligned}
P_{\text{safe}}=\arg\min_{P}\ & R(P)=\sum_{e\in P}\rho_e \\
\text{s.t.}\ & L(P)\le (1+\varepsilon)\,L(P_{\text{dist}}),
\end{aligned}
\end{equation}
where $\varepsilon$ is the allowable relative detour~\citep{ehrgott2005multicriteria}. We approximate this constraint via a weighted-sum sweep: for $\lambda\in\Lambda$,
\[
P(\lambda)
=\arg\min_{P}
\left(
\sum_{e\in P}\rho_e
+\lambda\sum_{e\in P}\ell_e
\right),
\]
and select among feasible candidates the route minimizing $R(P)$. Shortest paths are computed with Dijkstra’s algorithm~\citep{Dijkstra1959}.

\paragraph{Evaluation metrics.}
For each origin--destination pair, we report the relative length increase
\[
\Delta_L=\frac{L(P_{\text{safe}})-L(P_{\text{dist}})}{L(P_{\text{dist}})}
\]
and the relative risk reduction
\[
\Delta_R=\frac{R(P_{\text{dist}})-R(P_{\text{safe}})}{R(P_{\text{dist}})}.
\]
Pairs with $R(P_{\text{dist}})=0$ are excluded from $\Delta_R$ due to the undefined denominator. These metrics quantify the trade-off between distance and exposure-adjusted crash risk under bounded detours.

\section{Related work.}\label{sec:relatedw}
Prior work seeks to avoid conflating danger with demand by normalizing bicycle crashes by cyclist exposure~\citep{luecken2018}. City-scale studies show that exposure-normalized risk yields more
informative spatial patterns than raw crash counts and that finer temporal resolution improves inference, though persistent under-reporting in police records remains a challenge~\citep{Uijtdewilligen01092024}.
A central obstacle is obtaining reliable exposure: earlier work extrapolates city-wide volumes from sparse counters using learning-based models and multi-source features, while short-term measurement
campaigns improve predictions at unseen locations~\citep{Kaiser_Klein_Kaack_2025}. More recent efforts instead rely on street-segment datasets of measured bicycle volumes, enabling safety analyses 
without explicit exposure modeling~\citep{kaiser2025spatiotemporalgraphneuralnetwork}. At the network level, risk is typically defined as crashes per unit exposure on links, with attention to spatial
snapping, assignment of incidents to intersections, and integration of safety metrics into routing under convenience constraints~\citep{isprs-archives-XLIII-B4-2022-427-2022}. Intersection safety is 
repeatedly emphasized, with strong crash concentrations at junctions and the need to control for exposure when comparing locations or infrastructure types~\citep{futuretransp1030037}.

\begin{figure*}[!ht]
  \centering
  \includegraphics{figs/risk_3_panel.pdf}
  \caption{\textbf{Risk patterns and detailed inspection of a high-risk junction in Berlin.}
  (a) Network-wide cyclist risk map, where colors (\coolwarmbox) indicate log-scaled relative risk from low ({\color{TUblue}blue}) to high ({\color{TUred}red}).
  (b) Detail view of junction 2482 (Hermann-Hesse-Straße / Heinrich-Mann-Straße) with recorded accident locations.
  A total of 22 accidents occurred at this junction, predominantly involving a second vehicle, mostly cars (20 cases), with two involving other vehicle types.
  Most accidents were associated with vehicles turning into or crossing the road (20), while only two resulted from turning off the road.
  (c) Street-level view of the junction~\citep{streetview2024}, illustrating that the car lane crosses the bicycle lane.}
  \label{fig:risk_panel}
\end{figure*}

\section{Results}\label{sec:results}
\paragraph{Relative risk map.}
As outlined in the visual abstract (see \cref{fig:visual_app}), the relative risk of Strava bicycle segments and junctions in Berlin was estimated.
A section of the resulting risk map is shown in \cref{fig:risk_panel} (a).
Relative risk values range from 0.03 to 50.79 for road segments and from 0.03 to 6.42 for junctions.
To further investigate an exemplary location with elevated risk, we focus on junction 2482, which shows a comparatively high relative risk of 6.40.
The detailed view in \cref{fig:risk_panel} (b) shows all crashes assigned to this junction, the majority of which are located along the lane of Hermann-Hesse-Straße turning into Heinrich-Mann-Straße.
Analysis of the 22 crashes assigned to this junction reveals that all involved at least one additional vehicle besides a bicycle, predominantly cars (20 cases). 
Most crashes occurred during turning or crossing maneuvers (20 cases), while only a small number resulted from vehicles turning off the road (2 cases). 
To contextualize these findings, a street-level image of the junction from Google Street View \cite{streetview2024} in \cref{fig:risk_panel} (c) shows the car lanes crossing the bicycle lane at this location.

\paragraph{Safety-aware routing.}
To evaluate the routing algorithm, we sample $n=1000$ origin--destination pairs uniformly at random and compare shortest-distance routes with safety-aware alternatives~\citep{NateraOrozco2020}.

\begin{table}[ht]
\centering
\caption{Distance--risk trade-off under bounded detours for different junction-risk weights $\eta$.
Values are aggregated over all origin--destination pairs.
Medians are reported with interquartile ranges in parentheses.
$\Delta_L$ and $\Delta_R$ are reported on a relative scale, whereas $P(\Delta_R>0)$ is reported in percent.}
\label{tab:routing_tradeoff}
\begin{tabular}{@{}c c c c c@{}}
\toprule
$\eta$ & $\varepsilon$ & Med.\ $\Delta_L$ & Med.\ $\Delta_R$ & $P(\Delta_R>0)$ \\
\midrule
0.0 & 0.05 & 0.009 (0.026) & 0.246 (0.451) & 76.1 \\
    & 0.10 & 0.025 (0.042) & 0.377 (0.388) & 86.0 \\
    & 0.20 & 0.038 (0.072) & 0.425 (0.353) & 90.6 \\
\addlinespace
0.5 & 0.05 & 0.008 (0.026) & 0.208 (0.401) & 75.3 \\
    & 0.10 & 0.026 (0.046) & 0.331 (0.370) & 86.4 \\
    & 0.20 & 0.047 (0.089) & 0.404 (0.323) & 92.0 \\
\addlinespace
1.0 & 0.05 & 0.008 (0.026) & 0.185 (0.363) & 75.9 \\
    & 0.10 & 0.028 (0.047) & 0.305 (0.345) & 86.3 \\
    & 0.20 & 0.050 (0.086) & 0.378 (0.318) & 91.8 \\
\bottomrule
\end{tabular}
\end{table}

\cref{tab:routing_tradeoff} summarizes the trade-off between route length and exposure-adjusted crash risk under bounded detours. Safety-aware routing identifies feasible alternatives for all 
origin--destination pairs across detour budgets and junction-risk weights.

Allowing a 10\% detour reduces exposure-adjusted crash risk by 31--38\% in median, with over 86\% of routes achieving a risk reduction for all values of the junction-risk weight. Larger detours 
further increase these gains, reaching median reductions of 38--43\% at $\varepsilon=0.20$, while even small detours ($\varepsilon=0.05$) yield measurable reductions of 18--25\%. Across all detour
budgets, increasing $\eta$ is associated with lower median risk reductions.


\section{Discussion and Conclusion}\label{sec:conclusion}
% Use this section to briefly summarize the entire text. Highlight limitations and problems, but also make clear statements where they are possible and supported by the analysis. 
Our results indicate that substantial reductions in exposure-adjusted crash risk can be achieved with relatively small increases in route length. While higher junction-risk weights reduce the magnitude
of the estimated risk reduction, the distance--risk trade-off persists across all configurations, with absolute gains depending on the chosen weighting.

Overall, allowing a 10\% increase in route length yields median exposure-adjusted risk reductions of 31--38\% for the majority of routes, demonstrating that safety-aware routing can effectively trade 
modest detours for meaningful safety improvements.


We provide implementation details, hyperparameters, and supplementary material, available at \url{https://github.com/ytobiaz/data_literacy}.

\clearpage

\section*{Contribution Statement}
Explain here, in one sentence per person, what each group member contributed. For example, you could write: Max Mustermann collected and prepared data. Gabi Musterfrau and John Doe performed the data analysis. Jane Doe produced visualizations. All authors will jointly wrote the text of the report. Note that you, as a group, a collectively responsible for the report. Your contributions should be roughly equal in amount and difficulty.

% \section*{Notes} 
% Your entire report has a \textbf{hard page limit of 4 pages} excluding references and the contribution statement. (I.e. any pages beyond page 4 must only contain the contribution statement and references). Appendices are \emph{not} possible. But you can put additional material, like interactive visualizations or videos, on a githunb repo (use \href{https://github.com/pnkraemer/tueplots}{links} in your pdf to refer to them). Each report has to contain \textbf{at least three plots or visualizations}, and \textbf{cite at least two references}. More details about how to prepare the report, inclucing how to produce plots, cite correctly, and how to ideally structure your github repo, will be discussed in the lecture, where a rubric for the evaluation will also be provided.


\bibliography{bibliography}
\bibliographystyle{icml2025}

\end{document}

% This document was modified from the files available at https://icml.cc/Conferences/2025/AuthorInstructions
% the full copyright notice is available within the file icml2025.sty