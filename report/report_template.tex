%%%%%%%% DATA LITERACY 2025 LATEX PROJECT TEMPLATE FILE %%%%%%%%%%%%%%%%%
%%% Based on the 2025 ICML template, available at https://icml.cc/Conferences/2025/AuthorInstructions %%%

\documentclass{article}

% Recommended, but optional, packages for figures and better typesetting:
\usepackage{microtype}
\usepackage{graphicx}
\usepackage{subfigure}
\usepackage{booktabs} % for professional tables

\usepackage{tikz}
% Corporate Design of the University of Tübingen
% Primary Colors
\definecolor{TUred}{RGB}{165,30,55}
\definecolor{TUgold}{RGB}{180,160,105}
\definecolor{TUdark}{RGB}{50,65,75}
\definecolor{TUgray}{RGB}{175,179,183}

% Secondary Colors
\definecolor{TUdarkblue}{RGB}{65,90,140}
\definecolor{TUblue}{RGB}{0,105,170}
\definecolor{TUlightblue}{RGB}{80,170,200}
\definecolor{TUlightgreen}{RGB}{130,185,160}
\definecolor{TUgreen}{RGB}{125,165,75}
\definecolor{TUdarkgreen}{RGB}{50,110,30}
\definecolor{TUocre}{RGB}{200,80,60}
\definecolor{TUviolet}{RGB}{175,110,150}
\definecolor{TUmauve}{RGB}{180,160,150}
\definecolor{TUbeige}{RGB}{215,180,105}
\definecolor{TUorange}{RGB}{210,150,0}
\definecolor{TUbrown}{RGB}{145,105,70}
\definecolor{PNorange}{RGB}{255,153,51}

% hyperref makes hyperlinks in the resulting PDF.
% If your build breaks (sometimes temporarily if a hyperlink spans a page)
% please comment out the following usepackage line and replace
% \usepackage{icml2023} with \usepackage[nohyperref]{icml2023} above.
\usepackage{hyperref}


% Attempt to make hyperref and algorithmic work together better:
\newcommand{\theHalgorithm}{\arabic{algorithm}}

% command for plotting legend colorbar of risk map figure
\newcommand{\coolwarmbox}{%
  \tikz[baseline]{
    \begin{scope}[yshift=0.1ex]
      \draw[black!70]
        (0,0) rectangle (4em,1ex);
      \shade[
        shading=axis,
        shading angle=0,
        left color=TUblue,
        middle color= TUlightblue,
        right color=TUlightblue
      ]
      (0,0) rectangle (1em,1ex);
      \shade[
        shading=axis,
        shading angle=0,
        left color=TUlightblue,
        middle color=TUgray,
        right color=TUgray
      ]
      (0.9em,0) rectangle (2em,1ex);
      \shade[
        shading=axis,
        shading angle=0,
        left color=TUgray,
        middle color=PNorange,
        right color=PNorange
      ]
      (1.9em,0) rectangle (3em,1ex);
      \shade[
        shading=axis,
        shading angle=0,
        left color=PNorange,
        middle color=TUred,
        right color=TUred
      ]
      (2.9em,0) rectangle (4em,1ex);
    \end{scope}
  }%
}



\usepackage[accepted]{icml2025}

% For theorems and such
\usepackage{amsmath}
\usepackage{amssymb}
\usepackage{mathtools}
\usepackage{amsthm}
\usetikzlibrary{arrows.meta, positioning, calc}

% if you use cleveref..
\usepackage[capitalize,noabbrev]{cleveref}

% Todonotes is useful during development; simply uncomment the next line
%    and comment out the line below the next line to turn off comments
%\usepackage[disable,textsize=tiny]{todonotes}
\usepackage[textsize=tiny]{todonotes}


% The \icmltitle you define below is probably too long as a header.
% Therefore, a short form for the running title is supplied here:
\icmltitlerunning{Exposure-Adjusted Bicycle Crash Risk Estimation and Safer Routing in Berlin} % I think our title is not too long, hence we can use the same as for \icmltitle

\begin{document}

\twocolumn[
\icmltitle{Exposure-Adjusted Bicycle Crash Risk Estimation and Safer Routing in Berlin}

% It is OKAY to include author information, even for blind
% submissions: the style file will automatically remove it for you
% unless you've provided the [accepted] option to the icml2023
% package.

% List of affiliations: The first argument should be a (short)
% identifier you will use later to specify author affiliations
% Academic affiliations should list Department, University, City, Region, Country
% Industry affiliations should list Company, City, Region, Country

% You can specify symbols, otherwise they are numbered in order.
% Ideally, you should not use this facility. Affiliations will be numbered
% in order of appearance and this is the preferred way.
\icmlsetsymbol{equal}{*}

\begin{icmlauthorlist}
\icmlauthor{Eric Berger}{equal}
\icmlauthor{Edward Eichhorn}{equal}
\icmlauthor{Liaisan Faidrakhmanova}{equal}
\icmlauthor{Luise Grasl}{equal}
\icmlauthor{Tobias Schnarr}{equal}
\end{icmlauthorlist}

% fill in your matrikelnummer, email address, degree, for each group member
% \icmlaffiliation{first}{Matrikelnummer 7064584, MSc Machine Learning}
% \icmlaffiliation{second}{Matrikelnummer 12345678, MSc Quantitative Data Science}
% \icmlaffiliation{third}{Matrikelnummer 7320172, MSc Quantitative Data Science}
% \icmlaffiliation{fourth}{Matrikelnummer 7329274, MSc Quantitative Data Science}
% \icmlaffiliation{fith}{Matrikelnummer 7304640, MSc Quantitative Data Science}

% put your email addresses here. You can use initials to save space, 
% e.g. if you are called Max Mustermann, you can use \icmlcorrespondingauthor{MM}{max.mustermann@uni-tuebingen.de}
% DO USE YOUR UNIVERSITY EMAIL ADDRESS!
% \icmlcorrespondingauthor{EB}{eric.berger@student.uni-tuebingen.de} 
% \icmlcorrespondingauthor{EE}{first2.last2@uni-tuebingen.de}
% \icmlcorrespondingauthor{LF}{liaisan.faidrakhmanova@student.uni-tuebingen.de}
% \icmlcorrespondingauthor{LG}{luise.grasl@student.uni-tuebingen.de}
\icmlcorrespondingauthor{Tobias Schnarr}{tobias-marco.schnarr@student.uni-tuebingen.de}

% You may provide any keywords that you
% find helpful for describing your paper; these are used to populate
% the "keywords" metadata in the PDF but will not be shown in the document
\icmlkeywords{Machine Learning, ICML}

\vskip 0.3in
]

% this must go after the closing bracket ] following \twocolumn[ ...

% This command actually creates the footnote in the first column
% listing the affiliations and the copyright notice.
% The command takes one argument, which is text to display at the start of the footnote.
% The \icmlEqualContribution command is standard text for equal contribution.
% Remove it (just {}) if you do not need this facility.

%\printAffiliationsAndNotice{}  % leave blank if no need to mention equal contribution
\printAffiliationsAndNotice{\icmlEqualContribution} % otherwise use the standard text.

\begin{abstract}
Accurately estimating the risk of bicycle crashes at street level requires consideration of both crash counts and cyclist exposure. However, exposure data from official counting stations is unavailable for most streets. This makes it difficult to identify streets that are dangerous. We solve this by using Strava's bike 
trip data to estimate the relative crash risk across street segments and junctions in Berlin. We identify those with a higher or lower than expected occurrence of crashes, and enable a routing algorithm to suggest lower-risk routes.

\end{abstract}

\section{Introduction}\label{sec:intro}
\begin{figure*}[t]
  \centering
  \includegraphics{figs/map_3_panels.pdf}
  \caption{\textbf{Safety-aware routing pipeline for the Berlin network.}
  Panels (a--c) are zoomed in for readability; see \cref{sec:methods} for definitions and notation.
  (a) Police-recorded bicycle crashes in June 2021 ({\color{TUorange}points}) and street segments with measured cyclist exposure ({\color{TUgray}lines}).
  (b) Pooled segment-level relative risk estimated from all available data; high-risk segments in {\color{TUred}red} correspond to values above the 90th percentile of relative risk; {\color{TUdark}circles} mark junctions (degree $\ge 3$).
  (c) Shortest path ({\color{TUblue}blue}) versus a safer alternative ({\color{TUgreen}green}) selected to reduce cumulative relative risk under a route-length constraint. {\color{TUocre}Filled circle} and {\color{TUocre}cross} denote origin and destination, respectively.
  }
  \label{fig:visual_app}
\end{figure*}

Cycling is far from a safe endeavour. 92,882 bicycle crashes were recorded in 2024, including 441 fatalities---16\% of all traffic deaths that year ~\citep{Unfallatlas2025}. 
Yet it is rarely clear which streets are most dangerous and thus should be avoided by cyclists or made safer. 
Quantifying street-level danger is non-trivial because simple crash counts confound risk with exposure. 
Streets with high exposure, i.e. high numbers of cyclists, tend to accumulate more crashes even when per-cyclist risk is low~\citep{luecken2018}. Crashes must be normalised by cyclist counts; otherwise, dangerous streets can remain hidden in dense urban networks~\citep{Uijtdewilligen01092024}. 
Unfortunately, street-level cyclist counts are rare. Berlin, for example, provides hourly counts via official counting stations, but their limited coverage (20 stations for thousands of streets) makes them impractical for city-wide risk estimation~\citep{BerlinZaehlen_Fahrradbarometer}.
We address this problem by using bike trip counts from the fitness-tracking app Strava. These have been used to predict official bike counts~\citep{dadashova2020estimation}. We show that they can serve as a proxy for cyclist exposure and estimate for all segments and junctions in Berlin’s official cycling network relative risks (the ratio of observed to expected crashes).
Because Strava coverage can be sparse, we use Empirical Bayes smoothing for estimation~\citep{clayton1987empirical}. This stabilises estimates for low-exposure segments and junctions, and quantifies uncertainty. We also introduce a routing algorithm that finds substantially lower-risk routes under a route-length constraint.


\section{Data}\label{sec:data}
Multiple datasets were used for risk estimation. Crash counts were taken from the \emph{German Accident Atlas}~\citep{Unfallatlas2025}, which provides geodata of police-reported crashes where people were injured. We filtered the data to bicycle-related crashes within the city limits of Berlin. Cyclist exposure was approximated using the dataset by \citet{kaiser2025spatiotemporalgraphneuralnetwork}, 
which reports daily street-segment-level counts of bicycle trips recorded via the Strava app in Berlin from 2019 to 2023. Strava users are not representative of the general cycling population (they skew younger, male, and sport-oriented; \citealp{kaiser2025spatiotemporalgraphneuralnetwork}). Therefore, we assess potential bias by comparing segment-level count shares in 2023 with official bicycle 
counter data from the city of Berlin ~\citep{BerlinRadverkehrzaehlstellen2023} for the subset of segments where both Strava and official counts are available (Figure~\ref{fig:segment_share}). Count shares correlate strongly ($r = .61$) and are preserved in the Strava data. Segments on main streets where one can ride fast (e.g., Karl-Marx-Allee) are overrepresented in the Strava data, since those are more often tracked. Residential streets (e.g., Kollwitzstraße) are underrepresented, 
since slower, everyday cycling is less often tracked.
All datasets were combined into one dataframe and matched to the same street network. The network is represented as segments with associated monthly exposure counts. We map crashes to the network using nearest-segment assignment. Junctions are defined as nodes where at least three segments meet and crashes within a fixed radius are assigned to the nearest junction. Their exposure is derived from segment exposure (see \cref{sec:methods}). 
At monthly resolution, events are sparse: in a typical month, fewer than 4\% of segments and 3\% of junctions record at least one crash. We drop segments with zero recorded trips over at least one year and pool counts over the full period 2019--2023 for risk estimation. The final dataset comprises 4{,}335 segments, 2{,}862 junctions, and 15{,}396 recorded bicycle crashes. 

\begin{figure}[ht]
  \centering
  \includegraphics{figs/segment_share.pdf}
  \caption{Consistency check between official bicycle counts and Strava bike trips at the street-segment level (2023). Points show segment-wise shares of total annual counts.}
  \label{fig:segment_share}
\end{figure}



\section{Methods}\label{sec:methods}
\paragraph{Crash, exposure, and risk measures.}
For each month $t$, let $C_{s,t}$ and $E_{s,t}$ denote the number of police-recorded bicycle crashes and measured cyclist exposure on street segment $s$. Junction crashes $C_{v,t}$ are defined as crashes within a fixed radius of junction $v$. Because a traversal typically contributes exposure to two incident segments, we approximate junction exposure by the half-sum of incident segment exposures,
\[
E_{v,t}=\tfrac{1}{2}\sum_{s\in\mathcal I(v)} E_{s,t},
\]
a common approach when turning movements are unavailable~\citep{hakkert2002uses,WANG2020105838}.
For notational convenience, both street segments and junctions are indexed by a generic entity index $i$, with $A_{i,t}$ and $E_{i,t}$ denoting the corresponding crash and exposure quantities.

Under a no-special-risk baseline, crash incidence is assumed proportional to exposure, yielding the expected number of crashes and total sum of crashes for a given street-segment $i$ aggregated over time.
\[
\widehat{C}_{i}
= \frac{\sum_{i, t} C_{i, t}}{\sum_{i, t} E_{i, t}} \sum_{t} E_{i, t},
\quad
C_{i}=\sum_t C_{i, t}.
\]

The raw relative risk $r^{\text{raw}}_i$ is then given by $C_i/\widehat{C}_i$.

\paragraph{Empirical Bayes smoothing.}
Because many segments have low exposure and thus very small expected counts, the raw relative risk is highly variable. That is why we use Empirical Bayes smoothing to improve the risk estimates.
This methods shrinks low-exposure estimates toward a baseline, while high-exposure estimates change little.
Concretely, we assume a true relative risk  $r^{\text{true}}_i$ such that the observed count $C_i$ follows a Poisson model with 
\[
C_i \mid r^{\text{true}}_i \sim \text{Poisson}(\widehat{C}_i\,r^{\text{true}}_i).
\]
The Poisson distribution is natural for nonnegative event counts over a fixed time period under a baseline rate, and it yields $\mathbb{E}[C_i]=\widehat C_i$ when $r^{\text{true}}_i=1$. To allow heterogeneity in 
relative risk beyond this baseline, we place a Gamma prior on $r^{\text{true}}_i$ in the shape–rate parameterization,
\[
r^{\text{true}}_i \sim \text{Gamma}(\alpha,\alpha),
\]
which enforces $\mathbb{E}[r^{\text{true}}_i]=1$ and has variance $\mathrm{Var}(r^{\text{true}}_i)=1/\alpha$ controlling the amount of shrinkage. The Gamma prior is also conjugate 
to the Poisson likelihood, giving a closed-form posterior
\[
r^{\text{true}}_i \mid C_i,\widehat{C}_i \sim \text{Gamma}(C_i+\alpha,\;\widehat{C}_i+\alpha),
\]
so posterior inference is simple and numerically stable. We estimate $\alpha$ from the data using method of moments~\citep{Morris1983}, as
\[
\widehat{\alpha}
=
\frac{\sum_i \widehat{C}_i^{\,2}}
{\sum_i (C_i-\widehat{C}_i)^2 - \sum_i \widehat{C}_i}.
\]
and use the posterior mean
\[
\widehat{r}_i=\mathbb{E}[r^{\text{true}}_i\mid C_i,\widehat C_i]=\frac{C_i+\alpha}{\widehat C_i+\alpha}
\]
as the smoothed relative risk. For small $\widehat C_i$, $r_i$ is pulled toward 1, while for large $\widehat C_i$ it approaches the raw ratio $C_i/\widehat C_i$. Uncertainty is summarized by $(1-\delta=0.95)$ 
equal-tailed credible intervals from quantiles of the Gamma posterior. Since our goal is to identify segments or junctions whose risk deviates from the baseline, we first set the relative risk to 1 whenever its 
credible interval includes 1. For all remaining cases, we use a conservative deviation estimate: the upper credible limit if the interval lies entirely below 1, and the lower credible limit if it lies entirely above 1. 
We use these adjusted risks in all subsequent analyses.

\paragraph{Risk-weighted routing graph.}
Relative risk estimates are dimensionless and conditional on exposure. To obtain additive routing weights, we rescale relative risk by the pooled baseline crash rate,
\[
\bar{\lambda}=\frac{C_\cdot}{E_\cdot},
\qquad
C_\cdot=\sum_i C_i,\ \ E_\cdot=\sum_i E_i,
\]
yielding the routing weight $w_i=\bar{\lambda}\,\widehat{r}_i$.
We construct an undirected graph $G=(V,E)$ from the street network, where nodes correspond to segment endpoints and edges to street segments of length $\ell_e$. Each edge $e$ corresponds to a 
segment $s$ and inherits its weight, $w_e=w_s$. Junction identifiers and weights are mapped to nodes via spatial snapping in a projected coordinate system, producing a single risk-annotated 
network.

\paragraph{Safety-aware routing.}
We compare shortest-distance routes with alternatives that reduce estimated crash risk under a bounded detour. The length of a route $P$ is
\[
L(P)=\sum_{e\in P}\ell_e.
\]
To incorporate segment- and junction-level risk, the risk contribution of edge $e=(u,v)$ is defined as
\[
\rho_e = w_e + \frac{w_u+w_v}{2},
\]
where $w_u$ and $w_v$ denote junction routing weights (zero for non-junction nodes), yielding an additive surrogate for cumulative route risk.

For an origin--destination pair, the baseline route $P_{\text{dist}}$ minimizes $L(P)$. The safety-aware route is obtained by solving
\begin{equation*}
\begin{aligned}
P_{\text{safe}}&=\arg\min_{P}\ R(P)=\arg\min_{P}\sum_{e\in P}\rho_e \\
&\text{s.t.}\ L(P)\le (1+\varepsilon)\,L(P_{\text{dist}}),
\end{aligned}
\end{equation*}
where $\varepsilon$ is the allowable relative detour~\citep{ehrgott2005multicriteria}. We approximate this constraint using a weighted-sum sweep: for $\lambda\in\Lambda$,
\[
P(\lambda)
=\arg\min_{P}
\left(
\sum_{e\in P}\rho_e
+\lambda\sum_{e\in P}\ell_e
\right),
\]
and select the feasible route minimizing $R(P)$. Shortest paths are computed using Dijkstra’s algorithm~\citep{Dijkstra1959}.

\paragraph{Evaluation metrics.}
For each origin--destination pair, we report the relative length increase and relative risk reduction, as
\[
\Delta_L=\frac{L(P_{\text{safe}})-L(P_{\text{dist}})}{L(P_{\text{dist}})}
,\quad
\Delta_R=\frac{R(P_{\text{dist}})-R(P_{\text{safe}})}{R(P_{\text{dist}})}.
\]

We additionally report the expected number of avoided crashes, as $\Delta_C = R(P_{\text{dist}})-R(P_{\text{safe}})$.
Pairs with $R(P_{\text{dist}})=0$ are excluded from $\Delta_R$. 

\begin{figure*}[ht]
  \centering
  \includegraphics{figs/risk_3_panel.pdf}
  \caption{\textbf{Risk heatmap and detailed inspection of junction 2482.}
  Colors \coolwarmbox\space in panels (a)--(b) indicate $\log_{10}$-scaled risk values, ranging from low {\color{TUblue}(-2)} to high {\color{TUred}(2)}.
  (a) Section bike network with all computed road segments and junctions displayed.
  (b) Closer view of junction 2482 and the crashes (black dots) assigned to it
  (c) Street-level view of junction 2482 \citep{streetview2024}, providing visual context for the observed risk.
  }
  \label{fig:junction_details}
\end{figure*}



\section{Related Work}\label{sec:relatedw}
Previous studies on estimating street-level crash risk differ mainly in the proxy used for cyclist exposure and how exposure is modelled. For instance, \citet{isprs-archives-XLIII-B4-2022-427-2022} extrapolated bicycle volumes from motorised traffic data, but this does not accurately reflect actual cycling patterns. As an alternative to Strava-based exposure estimation, other studies have used crowdsourced GPS traces from less widely used tracking apps (e.g. BikeCitizens; \citealp{futuretransp1030037}) or city-wide counting events \citep{Uijtdewilligen01092024}. While these sources can approximate long-term station counts, they tend to be less stable as they rely on small user bases or special occasions.
Another approach predicts bicycle counts directly with models such as XGBoost and graph neural networks \citep{Kaiser_Klein_Kaack_2025, kaiser2025spatiotemporalgraphneuralnetwork}. While these methods can achieve strong predictive performance, they typically require rich, location-specific inputs and often additional calibration data, including manual counts on some streets. This is costly and rarely feasible at scale.
To address crash sparsity and overdispersion, prior work has also used Poisson–Gamma count models~\citep{luecken2018}. However, this was only for city-wide crash estimation and is not suited to estimating risk at segment level.
\section{Results}\label{sec:results}
We computed the relative risk values ($\widehat{r}_i$) for all street segments and junctions.
Due to high variance in the crash data, the shrinkage parameter is small ($\widehat{\alpha} = 0.129$), resulting in limited
regularization and a wide spread of $\widehat{r}_i$ estimates. These are shown in \cref{fig:junction_details}(a).
Most elements exhibit low risk: 64.4\% of segments and 69.1\% of junctions lie within the confidence bounds of the baseline ($\widehat{r}_i = 1$).
Values of $\widehat{r}_i<1$ occur for 17.6\% of segments and 25.8\% of junctions, while 17.9\% of segments and 4.9\% of junctions show elevated risk ($\widehat{r}_i>1$).
Risk values range from 0.03--50.79 for segments and 0.03--6.43 for junctions.
To verify that the method identifies high-risk locations, one such site was examined in detail.
At junction 2482 ($\widehat{r}_i = 6.43$), 22 crashes, most of them clustered along the right turning bike lane, occurred despite moderate traffic.
All involved at least one additional vehicle---20 of them were cars. This seems plausible, since---as shown in \cref{fig:junction_details}(c)---car lanes intersect the bicycle lane at this junction.

Furthermore, we evaluated the routing algorithm using 1,000 random origin--destination pairs, comparing the shortest-path baseline against safest alternatives with a detour constraint~\citep{NateraOrozco2020}.

\begin{table}[ht]
\centering
\small
\caption{Evaluation of the safety-aware routing algorithm with varying relative detour budgets ($\varepsilon$).
Values are aggregated over all origin--destination pairs and reported as medians with interquartile ranges.
$\Delta_L$ denotes the relative path length increase,
$\Delta_R$ the relative risk reduction,
and $\Delta_C$ the expected number of avoided crashes per 100{,}000 trips, reported as rounded integer counts.}
\label{tab:routing_tradeoff}
\begin{tabular}{@{}ccccc@{}}
\toprule
$\varepsilon$
& $\Delta_L\,(\mathrm{IQR})$
& $\Delta_R\,(\mathrm{IQR})$
& $\Delta_C\,(\mathrm{IQR})$
& $\Delta_R>0$ \\
\midrule
0.05 & 0.007\,(0.022) & 0.101\,(0.210) & 39\,(123)  & 0.767 \\
0.10 & 0.015\,(0.037) & 0.147\,(0.240) & 61\,(132)  & 0.841 \\
0.15 & 0.024\,(0.063) & 0.169\,(0.239) & 71\,(141)  & 0.873 \\
0.20 & 0.041\,(0.109) & 0.192\,(0.238) & 83\,(150)  & 0.901 \\
0.30 & 0.091\,(0.156) & 0.237\,(0.228) & 104\,(157) & 0.928 \\
0.40 & 0.137\,(0.188) & 0.262\,(0.230) & 112\,(170) & 0.944 \\
0.50 & 0.165\,(0.200) & 0.281\,(0.222) & 121\,(178) & 0.948 \\
\bottomrule
\end{tabular}
\end{table}

Table \ref{tab:routing_tradeoff} illustrates a clear trade-off between path length and safety, where increasing the detour budget $\varepsilon$ yields consistent safety improvements.
At the smallest budget of $\varepsilon = 0.05$, the algorithm achieves a median relative crash reduction ($\Delta_R$) of $10.1\%$, avoiding $39$ expected crashes per 100,000 trips ($\Delta_C$), while incurring a median distance increase ($\Delta_L$) of only $0.7\%$.
As the budget relaxes to $\varepsilon = 0.50$, the median safety gains scale to $\Delta_R = 28.1\%$ ($121$ expected avoided crashes), and the actual median path increase reaches $16.5\%$.
%This indicates that the most effective safety interventions are found within the first $15\%$ of detour distance.
While the interquartile ranges (IQR) for safety gains ($\Delta_R, \Delta_C$) remain relatively stable yet high, reflecting a large and persistent variability across routes, the IQR for path length ($\Delta_L$) expands significantly from $0.022$ to $0.200$, indicating that the distance cost required to achieve these improvements is highly sensitive to the specific layout of the route.
Furthermore, the likelihood of finding a safer route ($\Delta_R > 0$) remains high across all budgets, 
increasing from $76.7\%$ to $94.8\%$ as the search space expands.

\section{Discussion and Conclusion}\label{sec:conclusion}
We developed a method to estimate bicycle crash risk at the street-segment level using police-recorded 
crash counts and exposure data derived from Strava. These risk estimates were used to identify high-risk 
locations, where clear hazards could be observed. Based on these estimates, we implemented a safety-aware
routing algorithm that balances crash risk reduction against constraints on route length.
For modest detours of 5--15\%, the routing approach achieves substantial reductions in expected crash 
risk. The median risk decrease ranges from 10.1\% to 16.9\%, corresponding to an expected avoidance of 
39--71 crashes compared to the shortest-path route. However, the reduction in route-level risk strongly 
depends on the chosen origin and destination. Safe alternative routes are available for the majority of 
simulated trips (76.7\%--87.3\%). 
Several limitations must be acknowledged. Official bicycle crash counts are underreported and only 
include crashes involving injuries. In addition, the exposure data used in this study does not represent
the full cycling population due to the nature of Strava data. Finally, when computing junction exposure, 
we assume uniform flow and do not account for turning movements. 
Code and supplementary materials are available at \url{https://github.com/ytobiaz/data_literacy}.
\clearpage

\section*{Contribution Statement}
Eric Berger worked on visualizations and data correctness.
Edward Eichhorn worked on Empirical Bayes Smoothing and investigating the risk estimates.
Liaisan Faidrakhmanova worked on aggregating raw data and merging.
Luise Grasl worked on preprocessing and visualizations.
Tobias Schnarr worked on merging and routing.
All authors jointly wrote the text of the report.


% \section*{Notes} 
% Your entire report has a \textbf{hard page limit of 4 pages} excluding references and the contribution statement. (I.e. any pages beyond page 4 must only contain the contribution statement and references). Appendices are \emph{not} possible. But you can put additional material, like interactive visualizations or videos, on a githunb repo (use \href{https://github.com/pnkraemer/tueplots}{links} in your pdf to refer to them). Each report has to contain \textbf{at least three plots or visualizations}, and \textbf{cite at least two references}. More details about how to prepare the report, inclucing how to produce plots, cite correctly, and how to ideally structure your github repo, will be discussed in the lecture, where a rubric for the evaluation will also be provided.


\bibliography{bibliography}
\bibliographystyle{icml2025}

\end{document}

% This document was modified from the files available at https://icml.cc/Conferences/2025/AuthorInstructions
% the full copyright notice is available within the file icml2025.sty